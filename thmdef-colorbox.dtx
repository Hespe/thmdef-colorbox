% \iffalse meta-comment
%
% Copyright (C) 2025 by Christian Hespe
% --------------------------------------------------------------------
%
% This file may be distributed and/or modified under the conditions of
% the LaTeX Project Public License, version 1.3c.
% The license can be obtained from
% http://www.latex-project.org/lppl/lppl-1.3c.txt
%
% \fi
%
% \iffalse
%<*driver>
\ProvidesFile{thmdef-colorbox.dtx}
%</driver>
%<package>\NeedsTeXFormat{LaTeX2e}
%<package>\ProvidesPackage{thmdef-colorbox}
%<*package>
  [2025/09/06 v1.0 Colorbox attribute for thmtools]
%</package>
%
%<*driver>
\documentclass{ltxdoc}

\usepackage{amsthm}
\usepackage{thmtools}

\usepackage{listings}
\lstset{
  basicstyle=\ttfamily,
  columns=fullflexible,
  keepspaces=true,
  breakatwhitespace,
  gobble=4
}

\usepackage{tcolorbox}
\tcbuselibrary{skins}

\declaretheorem[colorbox]{theorem}
\declaretheorem[colorbox={sharp corners=all, colframe=blue}]{lemma}
\declaretheorem[colorbox={%
  enhanced jigsaw,
  colbacktitle=red,
  attach boxed title to bottom right
}]{corollary}
\declaretheoremstyle[
  notebraces="",
  headfont=\normalfont\scshape,
  notefont=\normalfont\itshape,
  bodyfont=\normalfont\ttfamily,
  colorbox={}
]{proposition}
\declaretheorem[style=proposition]{proposition}

\EnableCrossrefs
\CodelineIndex
\RecordChanges

\NewDocElement[idxtype={}, idxgroup={}]{Key}{key}
\setcounter{IndexColumns}{2}

\begin{document}
  \DocInput{thmdef-colorbox.dtx}
  \PrintChanges
  \PrintIndex
\end{document}
%</driver>
% \fi
%
% \CheckSum{0}
%
% \CharacterTable
%  {Upper-case    \A\B\C\D\E\F\G\H\I\J\K\L\M\N\O\P\Q\R\S\T\U\V\W\X\Y\Z
%   Lower-case    \a\b\c\d\e\f\g\h\i\j\k\l\m\n\o\p\q\r\s\t\u\v\w\x\y\z
%   Digits        \0\1\2\3\4\5\6\7\8\9
%   Exclamation   \!     Double quote  \"     Hash (number) \#
%   Dollar        \$     Percent       \%     Ampersand     \&
%   Acute accent  \'     Left paren    \(     Right paren   \)
%   Asterisk      \*     Plus          \+     Comma         \,
%   Minus         \-     Point         \.     Solidus       \/
%   Colon         \:     Semicolon     \;     Less than     \<
%   Equals        \=     Greater than  \>     Question mark \?
%   Commercial at \@     Left bracket  \[     Backslash     \\
%   Right bracket \]     Circumflex    \^     Underscore    \_
%   Grave accent  \`     Left brace    \{     Vertical bar  \|
%   Right brace   \}     Tilde         \~}
%
% \changes{v1.0}{2025/09/06}{Initial version}
%
% \GetFileInfo{thmdef-colorbox.dtx}
%
% \DoNotIndex{\RequirePackage}
% \DoNotIndex{\begin, \def, \end, \sbox}
% \DoNotIndex{\makeatletter, \makeatother}
% \DoNotIndex{\ignorespaces, \normalfont}
% \DoNotIndex{\thmt@envname, \thmt@trytwice}
%
% \title{The \textsf{thmdef-colorbox} package\thanks{This document
% corresponds to \textsf{thmdef-colorbox}~\fileversion, dated
% \filedate.}}
% \author{Christian Hespe}
%
% \maketitle
%
% \section{Introduction}
% The \textsf{thmdef-colorbox} package extends \textsf{thmtools} with
% the |colorbox| key, which wraps the theorem environment inside
% user-stylable boxes.
% Underneath, the boxes are drawn using \textsf{tcolorbox}, giving
% full access to \textsf{pgfkeys}-based styling.
% Compared to manually wrapping theorems inside boxes, this package
% extracts the theorem title for use within the |title| property of
% |tcolorbox|, enabling styles that separate title and body.
% It is therefore similar in scope to the \textsf{theorem} library of
% \textsf{tcolorbox} but remains compatible with \textsf{thmtools} and
% its machinery.
%
% \section{Usage}
% For basic usage, only the two packages%
% \begin{lstlisting}
%   \usepackage{amsthm}
%   \usepackage{thmtools}
% \end{lstlisting}
% have to be included.
% The \textsf{thmdef-colorbox} package itself will be loaded via the
% \textsf{thmtools} extension mechanism automatically once the
% |colorbox| key is used.
% In the document \emph{preamble}, declare a boxed theorem type using
% \lstinputlisting[linerange={41}]{thmdef-colorbox.dtx}
% with the |colorbox| key without further arguments.
% The main document body can then contain one or more theorems that
% get drawn as in the following snippet:\\
%
% \noindent\hfill
% \begin{minipage}{0.5\linewidth}
%   \begin{lstlisting}[breaklines, gobble=6]
%     \begin{theorem}[Euclid]
%       For every prime $p$, there is a larger prime $p'>p$.
%     \end{theorem}
%   \end{lstlisting}
% \end{minipage}
% \hfill
% \begin{minipage}{0.45\linewidth}
%   \begin{theorem}[Euclid]
%     For every prime $p$, there is a larger prime $p'>p$.
%   \end{theorem}
% \end{minipage}
% \hfill
%
% \subsection{Styling the theorem box}
% Since \textsf{tcolorbox} is used underneath, its capabilities can be
% used to achieve the different styles for theorems.
% The desired options can be passed directly to the |colorbox| key
% during declaration of the theorem in the document preamble:
% \lstinputlisting[linerange={42}]{thmdef-colorbox.dtx}
% In the main document, usage is otherwise identical.
% Note that subtitles using |\tcbsubtitle| and splitting the
% surrounding box with |\tcblower| are supported.\\
%
% \noindent\hfill
% \begin{minipage}{0.5\linewidth}
%   \begin{lstlisting}[breaklines, gobble=6]
%     \begin{lemma}
%       For every prime $p$,
%       \tcblower
%       there is a larger prime $p'>p$.
%     \end{lemma}
%   \end{lstlisting}
% \end{minipage}
% \hfill
% \begin{minipage}{0.45\linewidth}
%   \begin{lemma}
%     For every prime $p$,
%     \tcblower
%     there is a larger prime $p'>p$.
%   \end{lemma}
% \end{minipage}
% \hfill\\
%
% For some more advanced options of \textsf{tcolorbox}, it might be
% required to load additional libaries before setting the keys, e.g.
% for creating |breakable| environments or fancy themes.
% As an example, consider adding the title as a floating box at the
% bottom of the main theorem body.
% Since this requires the |enhanced jigsaw| theme, we first load the
% \textsf{skins} library of \textsf{tcolorbox} before declaring the
% theorem in the usual way:
% \lstinputlisting[linerange={39, 43-47}]{thmdef-colorbox.dtx}
% Note that manually loading the \textsf{tcolorbox} package might be
% necessary in this case to ensure that the |\tcbuselibrary| macro is
% available.\\
%
% \noindent\hfill
% \begin{minipage}{0.5\linewidth}
%   \begin{lstlisting}[breaklines, gobble=6]
%     \begin{corollary}
%       For every prime $p$, there is a larger prime $p'>p$.
%     \end{corollary}
%   \end{lstlisting}
% \end{minipage}
% \hfill
% \begin{minipage}{0.45\linewidth}
%   \begin{corollary}
%     For every prime $p$, there is a larger prime $p'>p$.
%   \end{corollary}
% \end{minipage}
% \hfill
%
% \subsection{Styling theorem title and contents}
% Compared to the box surrounding the theorem, styling the theorem
% itself is best left to \textsf{thmtools} instead of
% \textsf{tcolorbox}.
% The reason is that \textsf{thmtools} has access to the individual
% componenents of the theorem (e.g. name and note), while
% \textsf{tcolorbox} is getting the full title passed at once.
% For demonstration, consider the following style and theorem
% declaration:
% \lstinputlisting[linerange={48-55}]{thmdef-colorbox.dtx}
% Using options of \textsf{thmtools}, we can change the delimiters
% around the optional theorem note and adjust the fonts of head, note,
% and body separatly.\\
%
% \noindent\hfill
% \begin{minipage}{0.5\linewidth}
%   \begin{lstlisting}[breaklines, gobble=6]
%     \begin{proposition}[Euclid]
%       For every prime $p$, there is a larger prime $p'>p$.
%     \end{proposition}
%   \end{lstlisting}
% \end{minipage}
% \hfill
% \begin{minipage}{0.45\linewidth}
%   \begin{proposition}[Euclid]
%     For every prime $p$, there is a larger prime $p'>p$.
%   \end{proposition}
% \end{minipage}
% \hfill
%
% \StopEventually{}
%
% \section{Implementation}
% This package is implemented by taking advantage of the
% \textsf{thmtools} extension mechanism and works by patching an
% internal macro of \textsf{amsthm}.
%    \begin{macrocode}
\RequirePackage{amsthm}
\RequirePackage{thm-patch}
%    \end{macrocode}
% For patching, we rely on the |xapptocmd| macro of
% \textsf{regexpatch}, and the box drawing codes comes from the
% package \textsf{tcolorbox}.
%    \begin{macrocode}
\RequirePackage{regexpatch}
\RequirePackage{tcolorbox}
%    \end{macrocode}
% \begin{key}{colorbox}
% Starting with the main implementation, we register the |colorbox|
% key with \textsf{thmtools}, defaulting to an otherwise empty style.
%    \begin{macrocode}
\define@key{thmdef}{colorbox}[{}]{
  \thmt@trytwice{}{
%    \end{macrocode}
%
% \paragraph{Theorem head patch}
% The first change has to be applied before the actual theorem
% environment has been opened, otherwise the original theorem head
% has already been printed and thrown away.
% On the other hand, the title is only evaluated while opening the
% environment, therefore wrapping the theorem inside |tcolorbox| has
% to be deferred.
%
% \begin{macro}{\thmdef@colorbox@title}
%   To properly print the theorem header in the |tcolorbox|,
%   |\deferred@thm@head| has to be patched, first storing the
%   title in the |\thmdef@colorbox@title| macro and then throwing
%   away the title definition of the original macro.
%    \begin{macrocode}
    \addtotheorempreheadhook[\thmt@envname]{%
      \makeatletter
      \xapptocmd{\deferred@thm@head}{%
        \def\thmdef@colorbox@title{\normalfont##1}%
        \sbox\@labels{}%
        \ignorespaces
      }{}{\def\thmdef@colorbox@title{Patching failed}}%
      \makeatother
    }
%    \end{macrocode}
% \end{macro}
%
% \paragraph{tcolorbox wrapper}
% When the theorem environment has been opened, the theorem title is
% computed by \textsf{amsthm} and stored in
% \DescribeMacro{\thmdef@colorbox@title}|\thmdef@colorbox@title|
% by the patched macro above.
% We can therefore create the inner |tcolorbox| wrapper and assign the
% proper title.
% Finally, all options passed to the |colorbox| key are passed to the
% |tcolorbox| wrapper by |#1|.
%    \begin{macrocode}
    \addtotheorempostheadhook[\thmt@envname]{%
      \begin{tcolorbox}[title=\thmdef@colorbox@title, #1]%
    }
%    \end{macrocode}
%
% \paragraph{Closing the wrapper}
% Finally, the inner |tcolorbox| wrapper has to be closed before
% the outer theorem environment
%    \begin{macrocode}
    \addtotheoremprefoothook[\thmt@envname]{\end{tcolorbox}}
  }
}
%    \end{macrocode}
% which completes the definition of the |colorbox| key.
% \end{key}
%
% \Finale
